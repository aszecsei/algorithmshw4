\documentclass{article}

\usepackage[T1]{fontenc}

\usepackage{fancyhdr}
\usepackage{extramarks}
\usepackage{amsmath}
\usepackage{amsthm}
\usepackage{amsfonts}
\usepackage{tikz}
\usepackage{algorithm}
\usepackage{algpseudocode}
\usepackage{enumitem}

\usepackage[mono=false]{libertine}


\usetikzlibrary{automata,positioning}

%
% Basic Document Settings
%

\topmargin=-0.45in
\evensidemargin=0in
\oddsidemargin=0in
\textwidth=6.5in
\textheight=9.0in
\headsep=0.25in

\linespread{1.1}

\pagestyle{fancy}
\lhead{\hmwkAuthorName}
\chead{\hmwkClass: \hmwkTitle}
\rhead{\firstxmark}
\lfoot{\lastxmark}
\cfoot{\thepage}

\renewcommand\headrulewidth{0.4pt}
\renewcommand\footrulewidth{0.4pt}

\setlength\parindent{0pt}

%
% Create Problem Sections
%

\newcommand{\enterProblemHeader}[1]{
    \nobreak\extramarks{}{Problem \arabic{#1} continued on next page\ldots}\nobreak{}
    \nobreak\extramarks{Problem \arabic{#1} (continued)}{Problem \arabic{#1} continued on next page\ldots}\nobreak{}
}

\newcommand{\exitProblemHeader}[1]{
    \nobreak\extramarks{Problem \arabic{#1} (continued)}{Problem \arabic{#1} continued on next page\ldots}\nobreak{}
    \stepcounter{#1}
    \nobreak\extramarks{Problem \arabic{#1}}{}\nobreak{}
}

\setcounter{secnumdepth}{0}
\newcounter{partCounter}
\newcounter{homeworkProblemCounter}
\setcounter{homeworkProblemCounter}{1}
\nobreak\extramarks{Problem \arabic{homeworkProblemCounter}}{}\nobreak{}

%
% Homework Problem Environment
%
% This environment takes an optional argument. When given, it will adjust the
% problem counter. This is useful for when the problems given for your
% assignment aren't sequential. See the last 3 problems of this template for an
% example.
%
\newenvironment{homeworkProblem}[1][-1]{
    \ifnum#1>0
        \setcounter{homeworkProblemCounter}{#1}
    \fi
    \section{Problem \arabic{homeworkProblemCounter}}
    \setcounter{partCounter}{1}
    \enterProblemHeader{homeworkProblemCounter}
}{
    \exitProblemHeader{homeworkProblemCounter}
}

%
% Homework Details
%   - Title
%   - Due date
%   - Class
%   - Section/Time
%   - Instructor
%   - Author
%

\newcommand{\hmwkTitle}{Homework\ \#4}
\newcommand{\hmwkDueDate}{March 27, 2018}
\newcommand{\hmwkClass}{Design and Analysis of Algorithms}
\newcommand{\hmwkClassInstructor}{Professor Kasturi Varadarajan}
\newcommand{\hmwkAuthorName}{\textbf{Alic Szecsei}}

%
% Title Page
%

\title{
    \vspace{2in}
    \textmd{\textbf{\hmwkClass:\ \hmwkTitle}}\\
    \normalsize\vspace{0.1in}\small{Due\ in\ class\ on\ \hmwkDueDate}\\
    \vspace{0.1in}\large{\textit{\hmwkClassInstructor}}
    \vspace{3in}
}

\author{\hmwkAuthorName}
\date{}

\renewcommand{\part}[1]{\textbf{\large Part \Alph{partCounter}}\stepcounter{partCounter}\\}

%
% Various Helper Commands
%

% Useful for algorithms
\newcommand{\alg}[1]{\textsc{\bfseries \footnotesize #1}}

% For derivatives
\newcommand{\deriv}[1]{\frac{\mathrm{d}}{\mathrm{d}x} (#1)}

% For partial derivatives
\newcommand{\pderiv}[2]{\frac{\partial}{\partial #1} (#2)}

% Integral dx
\newcommand{\dx}{\mathrm{d}x}

% Alias for the Solution section header
\newcommand{\solution}{\textbf{\large Solution}}

% Probability commands: Expectation, Variance, Covariance, Bias
\newcommand{\E}{\mathrm{E}}
\newcommand{\Var}{\mathrm{Var}}
\newcommand{\Cov}{\mathrm{Cov}}
\newcommand{\Bias}{\mathrm{Bias}}

% Set from 1 to N
\newcommand{\XYZ}[1]{\left\{1,\ldots,{#1}\right\}}
\newcommand{\Break}{\textbf{break} }

\begin{document}

\maketitle

\pagebreak

\begin{homeworkProblem}

Consider the following randomized algorithm for generating biased random bits. The subroutine \alg{FairCoin} returns either 0 or 1 with equal probability; the random bits returned by \alg{FairCoin} are mutually independent.

\begin{algorithm}
	\begin{algorithmic}[1]
		\Function{OneInThree}{}
			\If{\Call{FairCoin}{} $= 0$}
				\State{\Return{0}}
			\Else
				\State{\Return{$1 - $\Call{OneInThree}{}}}
			\EndIf
		\EndFunction
	\end{algorithmic}
\end{algorithm}

\begin{enumerate}[label=(\alph*)]
	\item Prove that \alg{OneInThree} returns 1 with probability $\frac{1}{3}$.
	\item What is the \emph{exact} expected number of times that this algorithm calls \alg{FairCoin}?
	\item Now suppose you are \emph{given} a subroutine \alg{OneInThree} that generates a random bit that is equal to 1 with probability $\frac{1}{3}$. Describe a \alg{FairCoin} algorithm that returns either 0 or 1 with equal probability, using \alg{OneInThree} as your only source of randomness.
	\item What is the \emph{exact} expected number of times that your \alg{FairCoin} algorithm calls \alg{OneInThree}?
\end{enumerate}

\solution\\

\part

In order for \alg{OneInThree} to return 1, \alg{FairCoin} must return an odd number of 1s before returning a 0. We can begin to enumerate cases: first, that \alg{FairCoin} returns a 1 and then a 0 is $\frac{1}{2} \cdot \frac{1}{2} = \frac{1}{4}$. The probability that \alg{FairCoin} returns 3 1s and then a 0 is $\frac{1}{2} \cdot \frac{1}{2} \cdot \frac{1}{2} \cdot \frac{1}{2} = \frac{1}{16}$. The probability that \alg{FairCoin} returns 5 1s and then a 0 is $\frac{1}{2} \cdot \frac{1}{2} \cdot \frac{1}{2} \cdot \frac{1}{2} \cdot \frac{1}{2} \cdot \frac{1}{2} = \frac{1}{64}$, and so on. Our summation of all cases is then:
\begin{equation}
\frac{1}{4} + \frac{1}{16} + \frac{1}{64} + \ldots + \frac{1}{2^{2n}} = \sum_{k = 1}^{\infty} \frac{1}{2^{2k}}
\end{equation}

This is a geometric series with common ratio $r = \frac{1}{4}$; we can then use the formula $S = \frac{a_1}{1 - r}$ to determine the summation. Using this, we can see that $S = \frac{\frac{1}{4}}{1 - \frac{1}{4}} = \frac{\frac{1}{4}}{\frac{3}{4}} = \frac{1}{3}$. Therefore, \alg{OneInThree} has a $\frac{1}{3}$ probability of returning 1.\\

\part

Since \alg{OneInThree} might never terminate, our expected value of $T(n)$ is:
\begin{equation}
	E[T(n)] = \sum_{k = 1}^\infty k \cdot \text{Pr}[T(n) = k]
\end{equation}

In order for \alg{OneInThree} to terminate with exactly $k$ calls to \alg{FairCoin}, there must have been exactly $k-1$ times that \alg{FairCoin} returned 1 (and \alg{OneInThree} continued recursing), and 1 time it returned 0 (and \alg{OneInThree} stopped recursing). Thus, the combined probability for \alg{OneInThree} to terminate with exactly $k$ calls to \alg{FairCoin} is $\frac{1}{2^k}$. Plugging this back into our equation for $E[T(n)]$:

\begin{equation}
	\begin{split}
		E[T(n)] &= \sum_{k = 1}^\infty k \cdot \text{Pr}[T(n) = k]\\
		&= \sum_{k = 1}^\infty k \cdot \frac{1}{2^k}\\
		&= 2
	\end{split}
\end{equation}

\part

\begin{algorithm}
	\begin{algorithmic}[1]
		\Function{FairCoin}{}
			\State{$c_1 \gets \Call{OneInThree}{}$}
			\State{$c_2 \gets \Call{OneInThree}{}$}
			\If{$c_1 = 1$ and $c_2 = 1$}
				\State{\Return{\Call{FairCoin}{}}}
			\Else
				\If{$c_1 = 1$ or $c_2 = 1$}
					\State{\Return{$1$}}
				\Else
					\State{\Return{$0$}}
				\EndIf
			\EndIf
		\EndFunction
	\end{algorithmic}
\end{algorithm}

The idea for this algorithm is that we break the probability space into 3 partitions: first, the probability that both calls to \alg{OneInThree} return $1$ is $\frac{1}{9}$. The remaining probability is $\frac{8}{9}$; we further divide this into two partitions. The probability that both calls to \alg{OneInThree} returned $0$ is $\frac{4}{9}$; thus, the remaining probability (that only one call to \alg{OneInThree} returned $1$) is $\frac{4}{9}$.\\

Now, the probability that any call to \alg{FairCoin} which actually \emph{returns} will return 0 is clearly $\frac{1}{2}$; the same probability applies to any returning call to \alg{FairCoin} returning 1. Thus, \alg{FairCoin} has an equal probability of returning either 0 or 1.\\

\part

Since \alg{FairCoin} might never terminate, our expected value of $T(n)$ is:
\begin{equation}
	E[T(n)] = \sum_{k = 1}^\infty k \cdot \text{Pr}[T(n) = k]
\end{equation}

In order for \alg{FairCoin} to terminate with exactly $2k$ calls to \alg{OneInThree}, there must have been exactly $k-1$ times that both calls of \alg{OneInThree} returned 1 (and \alg{FairCoin} continued recursing), and 1 time at least one call returned 0 (and \alg{FairCoin} stopped recursing). Thus, the combined probability for \alg{FairCoin} to terminate with exactly $2k$ calls to \alg{FairCoin} is $\frac{1}{9^{k-1}} \cdot \frac{8}{9} = \frac{8}{9^k}$. Plugging this back into our equation for $E[T(n)]$:

\begin{equation}
	\begin{split}
		E[T(n)] &= \sum_{k = 1}^\infty 2k \cdot \text{Pr}[T(n) = k]\\
		&= \sum_{k = 1}^\infty 2k \cdot \frac{8}{9^k}\\
		&= \frac{9}{4}
	\end{split}
\end{equation}

\end{homeworkProblem}

\pagebreak

\begin{homeworkProblem}

Consider the following algorithm for finding the smallest element in an unsorted array:

\begin{algorithm}
	\begin{algorithmic}[1]
		\Function{RandomMin}{$A[1..n]$}
			\State{$min \gets \infty$}
			\For{$i \gets 1$ to $n$ in random order}
				\If{$A[i] < min$}
					\State{$min \gets A[i]$}\Comment{$(\star)$}
				\EndIf
			\EndFor
			\State{\Return{$min$}}
		\EndFunction
	\end{algorithmic}
\end{algorithm}

\begin{enumerate}[label=(\alph*)]
	\item In the worst case, how many times does \alg{RandomMin} execute line $(\star)$?
	\item What is the probability that line $(\star)$ is executed during the $i$th iteration of the for loop?
	\item What is the \emph{exact} expected number of executions of line $(\star)$?
\end{enumerate}

\solution\\

\part

In the worst case, \alg{RandomMin} retrieves the elements in descending order, so it must re-assign $min$ (in line $(\star)$) a total of $n$ times.\\

\part

If line $(\star)$ is executed during the $i$th iteration of the for loop, this means that the randomly selected element is smaller than the previous $i-1$ randomly selected elements. Essentially, we have a set of $i$ elements, and we want to know the probability that a specific element (the most recently added element) is the smallest element in that set. We have a single minimal element, and a $\frac{1}{i}$ chance of choosing that minimal element; therefore, the probability that the most recently added element is the minimal element of the set is $\frac{1}{i}$. We can thus conclude that the probability that line $(\star)$ is executed during the $i$th iteration of the for loop is also $\frac{1}{i}$.\\

\part

We know that line $(\star)$ has a $\frac{1}{i}$ probability of being executed during the $i$th iteration of the for loop, so the expected value at each iteration of the for loop is:

\begin{equation}
	E[T(n)]_{\text{iter}} = 0 \cdot \frac{i-1}{i} + 1 \cdot \frac{1}{i} = \frac{1}{i}
\end{equation}

We can sum all of these expected values together to determine the expected number of executions of line $(\star)$ during the entire algorithm:

\begin{equation}
	E[T(n)] = \sum_{k = 1}^{n} \frac{1}{k} = H_n
\end{equation}

\end{homeworkProblem}

\pagebreak

\begin{homeworkProblem}

Suppose we have a circular linked list of numbers, implemented as a pair of arrays, one storing the actual numbers and the other storing successor pointers. Specifically, let $X[1..n]$ be an array of $n$ distinct real numbers, and let $N[1..n]$ be an array of indices with the following property: If $X[i]$ is the largest element of $X$, then $X[N[i]]$ is the smallest element of $X$; otherwise, $X[N[i]]$ is the smallest among the set of elements in $X$ larger than $X[i]$. For example:

\begin{table}[h]
	\centering
		\begin{tabular}{r | *{9}{c} }
			$i$    &  1 &  2 &  3 &  4 &  5 &  6 &  7 &  8 &  9 \\
			\hline
			$X[i]$ & 83 & 54 & 16 & 31 & 45 & 99 & 78 & 62 & 27 \\
			\hline
			$N[i]$ &  6 &  8 &  9 &  5 &  2 &  3 &  1 &  7 &  4
		\end{tabular}
\end{table}

Describe and analyze a randomized algorithm that determines whether a given number $x$ appears in the array $X$ in $O(\sqrt{n})$ expected time. \emph{\textbf{Your algorithm may not modify the arrays $X$ and $N$.}}\\

\solution\\

\begin{algorithm}
	\begin{algorithmic}[1]
		\Function{RandomContains}{$X[1..n], N[1..n], x$}
			\State{$T \gets$ an array of $\sqrt{n}$ random numbers between 1 and $n$}
			\State{$hasLowerBound \gets False$} \Comment{We need to be able to handle the case where we do not select any elements less than $x$}
			\For{$i \gets 1, \sqrt{n}$}
				\If{$X[T[i]] < x$}
					\State{$hasLowerBound \gets True$}
					\Break
				\EndIf
			\EndFor
			\If{$hasLowerBound$}
				\State{$minDist \gets \infty$}
				\For{$i \gets 1, \sqrt{n}$}
					\If{$X[T[i]] < x$ and $x - X[T[i]] < minDist$}
						\State{$minDist \gets x - X[T[i]]$}
						\State{$l \gets T[i]$}
					\EndIf
				\EndFor
			\Else
				\State{$l \gets T[1]$}
				\For{$i \gets 2, \sqrt{n}$}
					\If{$X[T[i]] > X[l]$}
						\State{$l \gets T[i]$}
					\EndIf
				\EndFor
			\EndIf \Comment{We now have the ``closest'' element before $x$ (if it exists) on our linked list}
			\While{$True$}
				\If{$X[l] = x$}
					\State{\Return{$True$}}
				\Else
					\State{$prev \gets X[l]$}
					\State{$l \gets N[l]$}
					\If{$X[l] < prev$}
						\If{$hasLowerBound$}
							\State{\Return{$False$}} \Comment{We wrapped around before we found $x$}
						\Else
							\State{$hasLowerBound \gets True$} \Comment{We wrapped around, so we \emph{should} have a lower bound}
						\EndIf
					\EndIf
					\If{$X[l] > x$ and $hasLowerBound$}
						\State{\Return{$False$}} \Comment{We had a lower bound, but now we've passed where $x$ should have been}
					\EndIf
				\EndIf
			\EndWhile
		\EndFunction
	\end{algorithmic}
\end{algorithm}

Our algorithm selects $\sqrt{n}$ random pivots in the linked list, and searches them to find the greatest lower bound on our target value. If our target value is less than all of the pivots, it selects the largest pivot. We then walk through the linked list until we either find our target or exceed its value (with suitable edge cases handled if we need to wrap around).\\

First, we would like to show that the expected size $S$ of any of the $\sqrt{n}$ partitions our algorithm generates is, itself, $\sqrt{n}$. As an example, we will have 3 partitions of an array of length 9. We randomly select an element of the array $r$ and two pivots, $p_1$ and $p_2$; without loss of generality, assume $p_1 < p_2$. The probability that $r$ is between the two pivots is $\frac{1}{3}$, as the other two equiprobable alternatives are that $r < p_1$ and $r > p_2$. The probability that any value is in one of the partitions is $\frac{S}{L}$, where $L$ is the length of the array and $S$ is the partition size; thus, the expected length of each partition must be $\frac{1}{3} = \frac{S}{L} \Rightarrow \frac{1}{3} = \frac{S}{9} \Rightarrow S = 3$.\\

We can generalize this for any array length $L$ and number of partitions $P$; the expected partition size must be $\frac{L}{P}$. Since we have $\sqrt{n}$ partitions and an array of size $n$, our expected partition size is $\frac{n}{\sqrt{n}} = \sqrt{n}$.\\

Our algorithm uses $O(\sqrt{n})$ runtime to generate our pivot list and determine which partition contains $x$; it then linearly traverses the relevant partition. Since our expected partition size is \emph{also} $\sqrt{n}$, we can determine our final expected runtime to be $O(\sqrt{n})$.

\end{homeworkProblem}

\pagebreak

\begin{homeworkProblem}

A \emph{majority tree} is a complete binary tree with depth $n$, where every leaf is labeled either 0 or 1. The \emph{value} of a leaf is its label; the \emph{value} of any internal node is the majority of the values of its three children. Consider the problem of computing the value of the root of a majority tree, given the sequence of $3^n$ leaf labels as input. For example, if $n = 2$ and the leaves are labeled 1, 0, 0, 0, 1, 0, 1, 1, 1, the root has value 0.

\begin{figure}[h]
	\centering
		\includegraphics{images/majoritytree.png}
	\caption{A majority tree with depth $n = 2$.}
	\label{fig:majoritytree}
\end{figure}

\begin{enumerate}[label=(\alph*)]
	\item Prove that \emph{any} deterministic algorithm that computes the value of the root of a majority tree \emph{must} examine every leaf. \emph{[Hint: Consider the special case $n = 1$. Recurse.]}
	\item Describe and analyze a randomized algorithm that computes the value of the root in worst-case expected time $O(c^n)$ for some constant $c < 3$. \emph{[Hint: Consider the special case $n = 1$. Recurse.}
\end{enumerate}

\solution\\

\end{homeworkProblem}

\pagebreak

\begin{homeworkProblem}

Suppose you are given a graph $G$ with weighted edges, and your goal is to find a cut whose total weight (not just number of edges) is smallest.

\begin{enumerate}[label=(\alph*)]
	\item Describe an algorithm to select a random edge of $G$, where the probability of choosing edge $e$ is proportional to the weight of $e$.
	\item Prove that if you use the algorithm from part (a), instead of choosing edges uniformly at random, the probability that \alg{GuessMinCut} returns a minimum-weight is still $\Omega(1/n^2)$.
	\item What is the running time of your modified \alg{GuessMinCut} algorithm?
\end{enumerate}

\solution\\

\end{homeworkProblem}

\end{document}